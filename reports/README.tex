% Options for packages loaded elsewhere
\PassOptionsToPackage{unicode}{hyperref}
\PassOptionsToPackage{hyphens}{url}
%
\documentclass[
]{article}
\usepackage{lmodern}
\usepackage{amssymb,amsmath}
\usepackage{ifxetex,ifluatex}
\ifnum 0\ifxetex 1\fi\ifluatex 1\fi=0 % if pdftex
  \usepackage[T1]{fontenc}
  \usepackage[utf8]{inputenc}
  \usepackage{textcomp} % provide euro and other symbols
\else % if luatex or xetex
  \usepackage{unicode-math}
  \defaultfontfeatures{Scale=MatchLowercase}
  \defaultfontfeatures[\rmfamily]{Ligatures=TeX,Scale=1}
  \setmainfont[]{Times New Roman}
\fi
% Use upquote if available, for straight quotes in verbatim environments
\IfFileExists{upquote.sty}{\usepackage{upquote}}{}
\IfFileExists{microtype.sty}{% use microtype if available
  \usepackage[]{microtype}
  \UseMicrotypeSet[protrusion]{basicmath} % disable protrusion for tt fonts
}{}
\makeatletter
\@ifundefined{KOMAClassName}{% if non-KOMA class
  \IfFileExists{parskip.sty}{%
    \usepackage{parskip}
  }{% else
    \setlength{\parindent}{0pt}
    \setlength{\parskip}{6pt plus 2pt minus 1pt}}
}{% if KOMA class
  \KOMAoptions{parskip=half}}
\makeatother
\usepackage{xcolor}
\IfFileExists{xurl.sty}{\usepackage{xurl}}{} % add URL line breaks if available
\IfFileExists{bookmark.sty}{\usepackage{bookmark}}{\usepackage{hyperref}}
\hypersetup{
  pdftitle={README},
  hidelinks,
  pdfcreator={LaTeX via pandoc}}
\urlstyle{same} % disable monospaced font for URLs
\usepackage[margin=1in]{geometry}
\usepackage{color}
\usepackage{fancyvrb}
\newcommand{\VerbBar}{|}
\newcommand{\VERB}{\Verb[commandchars=\\\{\}]}
\DefineVerbatimEnvironment{Highlighting}{Verbatim}{commandchars=\\\{\}}
% Add ',fontsize=\small' for more characters per line
\usepackage{framed}
\definecolor{shadecolor}{RGB}{248,248,248}
\newenvironment{Shaded}{\begin{snugshade}}{\end{snugshade}}
\newcommand{\AlertTok}[1]{\textcolor[rgb]{0.94,0.16,0.16}{#1}}
\newcommand{\AnnotationTok}[1]{\textcolor[rgb]{0.56,0.35,0.01}{\textbf{\textit{#1}}}}
\newcommand{\AttributeTok}[1]{\textcolor[rgb]{0.77,0.63,0.00}{#1}}
\newcommand{\BaseNTok}[1]{\textcolor[rgb]{0.00,0.00,0.81}{#1}}
\newcommand{\BuiltInTok}[1]{#1}
\newcommand{\CharTok}[1]{\textcolor[rgb]{0.31,0.60,0.02}{#1}}
\newcommand{\CommentTok}[1]{\textcolor[rgb]{0.56,0.35,0.01}{\textit{#1}}}
\newcommand{\CommentVarTok}[1]{\textcolor[rgb]{0.56,0.35,0.01}{\textbf{\textit{#1}}}}
\newcommand{\ConstantTok}[1]{\textcolor[rgb]{0.00,0.00,0.00}{#1}}
\newcommand{\ControlFlowTok}[1]{\textcolor[rgb]{0.13,0.29,0.53}{\textbf{#1}}}
\newcommand{\DataTypeTok}[1]{\textcolor[rgb]{0.13,0.29,0.53}{#1}}
\newcommand{\DecValTok}[1]{\textcolor[rgb]{0.00,0.00,0.81}{#1}}
\newcommand{\DocumentationTok}[1]{\textcolor[rgb]{0.56,0.35,0.01}{\textbf{\textit{#1}}}}
\newcommand{\ErrorTok}[1]{\textcolor[rgb]{0.64,0.00,0.00}{\textbf{#1}}}
\newcommand{\ExtensionTok}[1]{#1}
\newcommand{\FloatTok}[1]{\textcolor[rgb]{0.00,0.00,0.81}{#1}}
\newcommand{\FunctionTok}[1]{\textcolor[rgb]{0.00,0.00,0.00}{#1}}
\newcommand{\ImportTok}[1]{#1}
\newcommand{\InformationTok}[1]{\textcolor[rgb]{0.56,0.35,0.01}{\textbf{\textit{#1}}}}
\newcommand{\KeywordTok}[1]{\textcolor[rgb]{0.13,0.29,0.53}{\textbf{#1}}}
\newcommand{\NormalTok}[1]{#1}
\newcommand{\OperatorTok}[1]{\textcolor[rgb]{0.81,0.36,0.00}{\textbf{#1}}}
\newcommand{\OtherTok}[1]{\textcolor[rgb]{0.56,0.35,0.01}{#1}}
\newcommand{\PreprocessorTok}[1]{\textcolor[rgb]{0.56,0.35,0.01}{\textit{#1}}}
\newcommand{\RegionMarkerTok}[1]{#1}
\newcommand{\SpecialCharTok}[1]{\textcolor[rgb]{0.00,0.00,0.00}{#1}}
\newcommand{\SpecialStringTok}[1]{\textcolor[rgb]{0.31,0.60,0.02}{#1}}
\newcommand{\StringTok}[1]{\textcolor[rgb]{0.31,0.60,0.02}{#1}}
\newcommand{\VariableTok}[1]{\textcolor[rgb]{0.00,0.00,0.00}{#1}}
\newcommand{\VerbatimStringTok}[1]{\textcolor[rgb]{0.31,0.60,0.02}{#1}}
\newcommand{\WarningTok}[1]{\textcolor[rgb]{0.56,0.35,0.01}{\textbf{\textit{#1}}}}
\usepackage{graphicx,grffile}
\makeatletter
\def\maxwidth{\ifdim\Gin@nat@width>\linewidth\linewidth\else\Gin@nat@width\fi}
\def\maxheight{\ifdim\Gin@nat@height>\textheight\textheight\else\Gin@nat@height\fi}
\makeatother
% Scale images if necessary, so that they will not overflow the page
% margins by default, and it is still possible to overwrite the defaults
% using explicit options in \includegraphics[width, height, ...]{}
\setkeys{Gin}{width=\maxwidth,height=\maxheight,keepaspectratio}
% Set default figure placement to htbp
\makeatletter
\def\fps@figure{htbp}
\makeatother
\setlength{\emergencystretch}{3em} % prevent overfull lines
\providecommand{\tightlist}{%
  \setlength{\itemsep}{0pt}\setlength{\parskip}{0pt}}
\setcounter{secnumdepth}{-\maxdimen} % remove section numbering

\title{README}
\author{}
\date{\vspace{-2.5em}}

\begin{document}
\maketitle

This document provides an overview of the \texttt{fsca} package. This
package was developed to calculate measures of syntactic complexity in
L2 French texts as part of a research project focusing on phraseological
complexity in learner language. For more details see Vandeweerd, Housen
and Paquot (in press).

Also see how to run a complete analysis in
\href{https://github.com/nvandeweerd/fsca/blob/main/example-analysis.md}{this
vignette}.

\hypertarget{licence}{%
\section{Licence}\label{licence}}

This package is distributed with an MIT license, which means that anyone
is free to use or modify the contents. In such cases, please the
following citation:

\begin{quote}
Vandeweerd, N. (in press). \emph{fsca: French syntactic complexity
analyzer.} International Journal of Learner Corpus Research.
\end{quote}

\hypertarget{installation}{%
\section{Installation}\label{installation}}

The \texttt{fsca} package can be installed from the github repository:

\begin{Shaded}
\begin{Highlighting}[]
\KeywordTok{install.packages}\NormalTok{(}\StringTok{"devtools"}\NormalTok{)}
\NormalTok{devtools}\OperatorTok{::}\KeywordTok{install_github}\NormalTok{(}\StringTok{"nvandeweerd/fsca"}\NormalTok{)}
\KeywordTok{library}\NormalTok{(fsca)}
\end{Highlighting}
\end{Shaded}

\hypertarget{overview}{%
\section{Overview}\label{overview}}

The package contains three functions:

\begin{itemize}
\tightlist
\item
  \texttt{getParse()}: a function to prepare dependency parsed texts for
  analysis
\item
  \texttt{getUnits()}: a function to extract syntactic units from
  dependency parsed texts
\item
  \texttt{getMeasures()}: a function to calculate measures of syntactic
  complexity from dependency parsed texts
\end{itemize}

To see the documentation for each function, call \texttt{help()}.

The package also contains a data file (\texttt{test.sents}) with the
example sentences found below. This can be loaded using \texttt{data()}.

\begin{Shaded}
\begin{Highlighting}[]
\KeywordTok{data}\NormalTok{(test.sents)}
\end{Highlighting}
\end{Shaded}

\hypertarget{preprocessing}{%
\section{Preprocessing}\label{preprocessing}}

The input of both \texttt{getUnits()} and \texttt{getMeasures()} is a
dependency parsed sentence in CONLL format. This function was written to
extract syntactic units from a corpus of L2 French texts in order to
calculate measures of syntactic complexity (see Vandeweerd, Housen, \&
Paquot, n.d.). The texts were POS-tagged with MElt Tagger (Denis \&
Sagot, 2012) and dependency parsed with Malt Parser (Nivre, Hall, \&
Nilsson, 2006).\footnote{On the basis of the dependency tags generated
  by Malt Parser, five new dependency labels were created for the
  purpose of calculating phraseological complexity measures: dobj:
  objects of verbs with the POS tag ``NC''; amod : noun modifiers with
  the POS tag ``ADJ''; advmod\_ADJ: modifiers of adjectives with the POS
  tag ``ADV''; advmod\_ADV: modifiers of adverbs with the POS tag
  ``ADV''; advmod\_VER: modifiers of verbs with the POS tag ``ADV''}

An example of a sentence in CONLL format is provided below:

\begin{verbatim}
[1] "C'est un point très important."
\end{verbatim}

\begin{verbatim}
##       TOKEN   POS.TT     LEMMA POS.MELT POSITION   DEP_TYPE DEP_ON
## 1        C'  PRO:DEM        ce      CLS        1        suj      2
## 2       est VER:pres      être        V        2       root      0
## 3        un  DET:ART        un      DET        3        det      4
## 4     point      NOM     point       NC        4        ats      2
## 5      très      ADV      très      ADV        5 advmod_ADJ      6
## 6 important      ADJ important      ADJ        6       amod      4
## 7         .     SENT         .    PONCT        7      ponct      2
\end{verbatim}

In this case, the main verb of the sentence is \emph{est}
(`is-3SG.PRES') and is labeled as the `root'. As the top level node of a
sentence, it is not dependent on any other word, hence the value of 0 in
the DEP\_ON column. Because \emph{est} is the second word of the
sentence, the position is 2 (indicated in the POSITION column). All
words which are directly dependent on \emph{est} have the value of 2 in
the DEP\_ON column. This includes the subject \emph{C'} (`it') as well
as the subject attribute \emph{point} (`point'). The sentence-final
period is also dependent on the root. The words which are directly
dependent on \emph{point} include the determiner \emph{un} (`a') and the
adjective \emph{important} (`important'), which is itself modified by
the adverb \emph{très} (`very'). In this way, every word in the sentence
is dependent on one and only one word.\footnote{For an overview of
  dependency parsing see Green (2011).}

\hypertarget{syntactic-structures}{%
\section{Syntactic structures}\label{syntactic-structures}}

The \texttt{getUnits()} function first searches for a list of node words
for a given unit (e.g.~nouns for noun phrases) and then extracts all of
the dependencies on each of the node words using the using the
\texttt{induced.subgraph()} function from the \texttt{igraph} package
(Csardi \& Nepusz, 2006). The definitions used for each unit are
provided below along with examples extracted from the corpus. Each
example provides the original sentence followed by the units which were
extracted. They have been simplified for readability by pasting the
tokens together.

\hypertarget{sentences}{%
\subsection{Sentences}\label{sentences}}

Following Lu (2010: 481) we defined a sentence as ``a group of words
delimited by one of the following punctuation marks that signal the end
of a sentence: period, question mark, exclamation mark, quotation mark
or ellipsis''. Because the CONLL input to the \texttt{getUnits()}
function is already segmented into sentences, no additional query is
required.

\hypertarget{clauses}{%
\subsection{Clauses}\label{clauses}}

Clauses are defined as structures with a subject and a finite verb
(Hunt, 1965). After identifying dependent clauses and T-units, clauses
include all T-units as well as all subordinate clauses emedded within
the T-units in a given sentence.

\hypertarget{dep-clauses}{%
\subsection{Dependent clauses}\label{dep-clauses}}

Dependent clauses are clauses which are semantically and/or structurally
dependent on a super-ordinate syntactic structure. They include nominal
clauses, adverbial clauses and adjectival clauses (Hunt, 1965; Lu,
2010). They must contain a finite verb and a subject.

\hypertarget{nomclause}{%
\subsubsection{Nominal clauses}\label{nomclause}}

Nominal clauses are extracted using subordinate conjunctions
(e.g.~\emph{que}, `which') as the main node.

\begin{verbatim}
[1] "Ils prétendent qu'il est impossible de rééduquer un tel jeune criminel."
\end{verbatim}

\begin{Shaded}
\begin{Highlighting}[]
\KeywordTok{getUnits}\NormalTok{(test.sents[[}\StringTok{"b.208.1"}\NormalTok{]], }
         \DataTypeTok{what =} \StringTok{"tokens"}\NormalTok{, }
         \DataTypeTok{units =} \KeywordTok{c}\NormalTok{(}\StringTok{"DEP_CLAUSES"}\NormalTok{), }
         \DataTypeTok{paste.tokens =} \OtherTok{TRUE}\NormalTok{)}
\CommentTok{## $DEP_CLAUSES}
\CommentTok{## $DEP_CLAUSES[[1]]}
\CommentTok{## [1] "qu' il est impossible de rééduquer un tel jeune criminel"}
\end{Highlighting}
\end{Shaded}

\hypertarget{advclause}{%
\subsubsection{Adverbial clauses}\label{advclause}}

Adverbial clauses are also extracted using subordinate conjunctions as
the main node.

\begin{verbatim}
[1] "Quand les lois sont contre le droit, il n'y a qu'une héroïque façon de protester contre elles : les violer (Hugo)."
\end{verbatim}

\begin{Shaded}
\begin{Highlighting}[]
\KeywordTok{getUnits}\NormalTok{(test.sents[[}\StringTok{"b.347.1"}\NormalTok{]], }
         \DataTypeTok{what =} \StringTok{"tokens"}\NormalTok{, }
         \DataTypeTok{units =} \KeywordTok{c}\NormalTok{(}\StringTok{"DEP_CLAUSES"}\NormalTok{), }
         \DataTypeTok{paste.tokens =} \OtherTok{TRUE}\NormalTok{)}
\CommentTok{## $DEP_CLAUSES}
\CommentTok{## $DEP_CLAUSES[[1]]}
\CommentTok{## [1] "Quand les lois sont contre le droit"}
\end{Highlighting}
\end{Shaded}

\hypertarget{adjclause}{%
\subsubsection{Adjectival clauses}\label{adjclause}}

Adjectival clauses are extracted using pronouns and finite verbs that
modify a noun as the main nodes (e.g.~\emph{durent} and \emph{arrivent}
in the example below).

\begin{verbatim}
[1] "Des procès qui durent des années, des déclarations d'impôts inhumainement longues et compliquées, un gouvernement qui n'arrive pas à prendre forme..."
\end{verbatim}

\begin{Shaded}
\begin{Highlighting}[]
\KeywordTok{getUnits}\NormalTok{(test.sents[[}\StringTok{"b.110.1"}\NormalTok{]], }
         \DataTypeTok{what =} \StringTok{"tokens"}\NormalTok{, }
         \DataTypeTok{units =} \KeywordTok{c}\NormalTok{(}\StringTok{"DEP_CLAUSES"}\NormalTok{), }
         \DataTypeTok{paste.tokens =} \OtherTok{TRUE}\NormalTok{)}
\CommentTok{## $DEP_CLAUSES}
\CommentTok{## $DEP_CLAUSES[[1]]}
\CommentTok{## [1] "qui durent des années"}
\CommentTok{## }
\CommentTok{## $DEP_CLAUSES[[2]]}
\CommentTok{## [1] "qui n' arrive pas à prendre forme"}
\end{Highlighting}
\end{Shaded}

\hypertarget{special-cases}{%
\subsubsection{Special cases}\label{special-cases}}

Clauses of the type `il y a' are considered dependent clauses only if
\emph{a} is directly dependent on a finite verb or if there is a finite
verb dependent on \emph{a}. This means that adverbial clauses which
function like `ago' in English (e.g.~\emph{il y a deux ans\ldots{}};
`two years ago\ldots{}') are captured as dependent clauses but simple
declaratives (e.g.~\emph{il y a une maison}; `there is a house') are
not.

\begin{verbatim}
[1] "Il y a quelques siècles, les empereurs pourraient modifier les règles selon leur volonté."
\end{verbatim}

\begin{Shaded}
\begin{Highlighting}[]
\KeywordTok{getUnits}\NormalTok{(test.sents[[}\StringTok{"a.65.1"}\NormalTok{]], }
         \DataTypeTok{what =} \StringTok{"tokens"}\NormalTok{, }
         \DataTypeTok{units =} \KeywordTok{c}\NormalTok{(}\StringTok{"DEP_CLAUSES"}\NormalTok{), }
         \DataTypeTok{paste.tokens =} \OtherTok{TRUE}\NormalTok{)}
\CommentTok{## $DEP_CLAUSES}
\CommentTok{## $DEP_CLAUSES[[1]]}
\CommentTok{## [1] "Il y a quelques siècles"}
\end{Highlighting}
\end{Shaded}

\begin{verbatim}
[1] "Ensuite, il y a des délits dus aux problèmes personnels survenus à la maison ou à l'école."
\end{verbatim}

\begin{Shaded}
\begin{Highlighting}[]
\KeywordTok{getUnits}\NormalTok{(test.sents[[}\StringTok{"a.57.1"}\NormalTok{]], }
         \DataTypeTok{what =} \StringTok{"tokens"}\NormalTok{, }
         \DataTypeTok{units =} \KeywordTok{c}\NormalTok{(}\StringTok{"DEP_CLAUSES"}\NormalTok{), }
         \DataTypeTok{paste.tokens =} \OtherTok{TRUE}\NormalTok{)}
\CommentTok{## $DEP_CLAUSES}
\CommentTok{## $DEP_CLAUSES[[1]]}
\CommentTok{## [1] "NA"}
\end{Highlighting}
\end{Shaded}

Direct interrogatives in the form \emph{est-ce que} are not considered
the head of subordinate clauses. Rather, the head of a clause is the
finite verb dominated by \emph{est} as in interrogatives formed by
inversion.

\begin{verbatim}
[1] "Est-ce que les règles sont nécessaires?"
\end{verbatim}

\begin{Shaded}
\begin{Highlighting}[]
\KeywordTok{getUnits}\NormalTok{(test.sents[[}\StringTok{"a.17.1"}\NormalTok{]], }
         \DataTypeTok{what =} \StringTok{"tokens"}\NormalTok{, }
         \DataTypeTok{units =} \KeywordTok{c}\NormalTok{(}\StringTok{"CLAUSES"}\NormalTok{, }\StringTok{"DEP_CLAUSES"}\NormalTok{), }
         \DataTypeTok{paste.tokens =} \OtherTok{TRUE}\NormalTok{)}
\CommentTok{## $CLAUSES}
\CommentTok{## $CLAUSES[[1]]}
\CommentTok{## [1] "Est -ce que les règles sont nécessaires"}
\CommentTok{## }
\CommentTok{## }
\CommentTok{## $DEP_CLAUSES}
\CommentTok{## $DEP_CLAUSES[[1]]}
\CommentTok{## [1] "NA"}
\end{Highlighting}
\end{Shaded}

Citations or reported speech enclosed with French guillmets («»), single
or double quotation marks are also considered subordinate clauses.

\begin{verbatim}
[1] "Il est possible que la langue disparaisse après quelques générations, car, comme dit le professeur Roegiest, professeur des langues romanes et linguiste de l'espagnol à l'université de Gand, « Une langue qui a perdu son prestige est souvent condamnée à mort »."
\end{verbatim}

\begin{Shaded}
\begin{Highlighting}[]
\KeywordTok{getUnits}\NormalTok{(test.sents[[}\StringTok{"b.274.1"}\NormalTok{]], }
         \DataTypeTok{what =} \StringTok{"tokens"}\NormalTok{, }
         \DataTypeTok{units =} \KeywordTok{c}\NormalTok{(}\StringTok{"DEP_CLAUSES"}\NormalTok{), }
         \DataTypeTok{paste.tokens =} \OtherTok{TRUE}\NormalTok{)}
\CommentTok{## $DEP_CLAUSES}
\CommentTok{## $DEP_CLAUSES[[1]]}
\CommentTok{## [1] "que la langue disparaisse après quelques générations"}
\CommentTok{## }
\CommentTok{## $DEP_CLAUSES[[2]]}
\CommentTok{## [1] "comme dit le professeur Roegiest professeur des langues romanes et linguiste de l' espagnol à l' université de Gand Une langue qui a perdu son prestige"}
\CommentTok{## }
\CommentTok{## $DEP_CLAUSES[[3]]}
\CommentTok{## [1] "Une langue qui a perdu son prestige est souvent condamnée à mort"}
\CommentTok{## }
\CommentTok{## $DEP_CLAUSES[[4]]}
\CommentTok{## [1] "qui a perdu son prestige"}
\end{Highlighting}
\end{Shaded}

\hypertarget{coordinated-clauses}{%
\subsection{Coordinated clauses}\label{coordinated-clauses}}

Coordinated clauses are clauses which are not semantically and/or
structurally dependent on a super-ordinate syntactic structure but are
conjoined to one or more clauses of syntactically equal status. They may
be joined by a coordinating conjunction (e.g.~\emph{et}; `and'),
punctuation (e.g.~semi-colon, colon, comma) or by juxtaposition and must
contain both a subject and a finite verb. They are extracted from the
root nodes of finite verbs which are not themselves dependent on
subordinate conjunctions or pronouns (to exclude dependent clauses).

\begin{verbatim}
[1] "Dans la prison, il n'ont plus d'éducation, il ne voient que des criminels et ils doivent devenir des adultes en nulle temps."
\end{verbatim}

\begin{Shaded}
\begin{Highlighting}[]
\KeywordTok{getUnits}\NormalTok{(test.sents[[}\StringTok{"a.90.1"}\NormalTok{]], }
         \DataTypeTok{what =} \StringTok{"tokens"}\NormalTok{, }
         \DataTypeTok{units =} \KeywordTok{c}\NormalTok{(}\StringTok{"CO_CLAUSES"}\NormalTok{), }
         \DataTypeTok{paste.tokens =} \OtherTok{TRUE}\NormalTok{)}
\CommentTok{## $CO_CLAUSES}
\CommentTok{## $CO_CLAUSES[[1]]}
\CommentTok{## [1] "Dans la prison il n' ont plus d' éducation"}
\CommentTok{## }
\CommentTok{## $CO_CLAUSES[[2]]}
\CommentTok{## [1] "il ne voient que des criminels"}
\CommentTok{## }
\CommentTok{## $CO_CLAUSES[[3]]}
\CommentTok{## [1] "et ils doivent devenir des adultes en nulle temps"}
\end{Highlighting}
\end{Shaded}

\hypertarget{tunits}{%
\subsection{T-units}\label{tunits}}

We use Hunt's (1970: 199) definition of a t-unit as ``one main clause
plus any subordinate clause or non-clausal structure that is attached to
or embedded in it''. Identifying t-units therefore depends on the
identification of coordinated clauses since a sentence can only contain
multiple t-units if it contains multiple coordinated clauses. A sentence
that does not contain any coordinated clauses simply has one t-unit,
provided it has at least one finite verb. Consistent with Hunt (1965),
we do not classify sentence fragments (clauses without a finite verb) as
t-units. Therefore, if a sentence has no coordinated clauses (and one
finite verb) it has one t-unit. All additional coordinated clauses
within a sentence are separate t-units. The three coordinated clauses in
the sentence above therefore account for three t-units.

\begin{verbatim}
[1] "Dans la prison, il n'ont plus d'éducation, il ne voient que des criminels et ils doivent devenir des adultes en nulle temps."
\end{verbatim}

\begin{Shaded}
\begin{Highlighting}[]
\KeywordTok{getUnits}\NormalTok{(test.sents[[}\StringTok{"a.90.1"}\NormalTok{]], }
         \DataTypeTok{what =} \StringTok{"number"}\NormalTok{, }
         \DataTypeTok{units =} \KeywordTok{c}\NormalTok{(}\StringTok{"CO_CLAUSES"}\NormalTok{,}\StringTok{"T_UNITS"}\NormalTok{))}
\CommentTok{## CO_CLAUSES    T_UNITS }
\CommentTok{##          3          3}
\end{Highlighting}
\end{Shaded}

\hypertarget{noun-phrases}{%
\subsection{Noun phrases}\label{noun-phrases}}

We use Lu's (2010) definition of a noun phrase as a \emph{complex
nominal} (see Cooper, 1976) which includes: nouns plus adjective(s),
possessive(s), prepositional phrase(s), relative clause(s),
participle(s), or appositive(s), nominal clause(s). We also include
words (nouns, adverbs and pronouns) that are have a determiner
(e.g.~\emph{une maison} `a house'; \emph{cet autre} `this other') in our
definition. Following Lu (2010) and Cooper (1976), we also include
gerunds and infinitives in subject position. The root nodes of noun
phrases therefore include (within each t-unit) common nouns and proper
nouns as well as gerunds and non-finite verbs when they are dependent on
a finite verb and have a subject dependency label.

\begin{verbatim}
[1] "La question sur une nouvelle réforme de l'État est un grand problème qui se pose aujourd'hui en Belgique."
\end{verbatim}

\begin{Shaded}
\begin{Highlighting}[]
\KeywordTok{getUnits}\NormalTok{(test.sents[[}\StringTok{"b.46.1"}\NormalTok{]], }
         \DataTypeTok{what =} \StringTok{"tokens"}\NormalTok{, }
         \DataTypeTok{units =} \KeywordTok{c}\NormalTok{(}\StringTok{"NOUN_PHRASES"}\NormalTok{), }
         \DataTypeTok{paste.tokens =} \OtherTok{TRUE}\NormalTok{)}
\CommentTok{## $NOUN_PHRASES}
\CommentTok{## $NOUN_PHRASES[[1]]}
\CommentTok{## [1] "La question sur une nouvelle réforme de l' État"}
\CommentTok{## }
\CommentTok{## $NOUN_PHRASES[[2]]}
\CommentTok{## [1] "une nouvelle réforme de l' État"}
\CommentTok{## }
\CommentTok{## $NOUN_PHRASES[[3]]}
\CommentTok{## [1] "l' État"}
\CommentTok{## }
\CommentTok{## $NOUN_PHRASES[[4]]}
\CommentTok{## [1] "un grand problème qui se pose aujourd' hui en Belgique"}
\CommentTok{## }
\CommentTok{## $NOUN_PHRASES[[5]]}
\CommentTok{## [1] "La question"}
\CommentTok{## }
\CommentTok{## $NOUN_PHRASES[[6]]}
\CommentTok{## [1] "une nouvelle réforme"}
\CommentTok{## }
\CommentTok{## $NOUN_PHRASES[[7]]}
\CommentTok{## [1] "un grand problème"}
\end{Highlighting}
\end{Shaded}

\hypertarget{verb-phrases}{%
\subsection{Verb phrases}\label{verb-phrases}}

As in Lu (2010) we count both finite and non-finite verb phrases. These
are extracted by taking all words which are dependent on a verb within a
t-unit. Auxiliary verbs do not constitute their own verb phrase but are
considered part of the main verb they modify. However, verb phrases with
modal verbs are considered separate verb phrases. After extracting the
dependents of each verb node, the units are cleaned by removing subjects
and pre-verbal modifiers. When two verbs are coordinated they are also
considered a singular verb phrase (e.g.~\emph{ne sont pas d'accord et
présentent de solutions différents}; `do not agree and present different
solutions').

\begin{verbatim}
[1] "Dotée d'un éventail de mots et d'expressions, la parole constitue le moyen de communication par excellence."
\end{verbatim}

\begin{Shaded}
\begin{Highlighting}[]
\KeywordTok{getUnits}\NormalTok{(test.sents[[}\StringTok{"b.124.1"}\NormalTok{]], }
         \DataTypeTok{what =} \StringTok{"tokens"}\NormalTok{, }
         \DataTypeTok{units =} \KeywordTok{c}\NormalTok{(}\StringTok{"VERB_PHRASES"}\NormalTok{), }
         \DataTypeTok{paste.tokens =} \OtherTok{TRUE}\NormalTok{)}
\CommentTok{## $VERB_PHRASES}
\CommentTok{## $VERB_PHRASES[[1]]}
\CommentTok{## [1] "Dotée d' un éventail de mots et d' expressions"}
\CommentTok{## }
\CommentTok{## $VERB_PHRASES[[2]]}
\CommentTok{## [1] "constitue le moyen de communication par excellence"}
\end{Highlighting}
\end{Shaded}

\hypertarget{calculating-mesures}{%
\section{Calculating mesures}\label{calculating-mesures}}

The function \texttt{getMeasures()} can be used to get several measures
of syntactic complexity from a list of sentences. The example below
calculates these measures for the sentences used above.

\begin{verbatim}
 [1] "C'est un point très important."                                                                                                                                                                                                                                      
 [2] "Ils prétendent qu'il est impossible de rééduquer un tel jeune criminel."                                                                                                                                                                                             
 [3] "Quand les lois sont contre le droit, il n'y a qu'une héroïque façon de protester contre elles : les violer (Hugo)."                                                                                                                                                  
 [4] "Des procès qui durent des années, des déclarations d'impôts inhumainement longues et compliquées, un gouvernement qui n'arrive pas à prendre forme..."                                                                                                               
 [5] "Il y a quelques siècles, les empereurs pourraient modifier les règles selon leur volonté."                                                                                                                                                                           
 [6] "Ensuite, il y a des délits dus aux problèmes personnels survenus à la maison ou à l'école."                                                                                                                                                                          
 [7] "Est-ce que les règles sont nécessaires?"                                                                                                                                                                                                                             
 [8] "Il est possible que la langue disparaisse après quelques générations, car, comme dit le professeur Roegiest, professeur des langues romanes et linguiste de l'espagnol à l'université de Gand, « Une langue qui a perdu son prestige est souvent condamnée à mort »."
 [9] "Dans la prison, il n'ont plus d'éducation, il ne voient que des criminels et ils doivent devenir des adultes en nulle temps."                                                                                                                                        
[10] "La question sur une nouvelle réforme de l'État est un grand problème qui se pose aujourd'hui en Belgique."                                                                                                                                                           
[11] "Dotée d'un éventail de mots et d'expressions, la parole constitue le moyen de communication par excellence."                                                                                                                                                         
\end{verbatim}

\begin{Shaded}
\begin{Highlighting}[]
\KeywordTok{getMeasures}\NormalTok{(examples)}
\CommentTok{## $MLS}
\CommentTok{## [1] 18.73}
\CommentTok{## }
\CommentTok{## $DIVS}
\CommentTok{## [1] 9.84}
\CommentTok{## }
\CommentTok{## $T_S}
\CommentTok{## [1] 1.18}
\CommentTok{## }
\CommentTok{## $MLT}
\CommentTok{## [1] 15.85}
\CommentTok{## }
\CommentTok{## $DIVT}
\CommentTok{## [1] 9.93}
\CommentTok{## }
\CommentTok{## $C_T}
\CommentTok{## [1] 1.77}
\CommentTok{## }
\CommentTok{## $MLC}
\CommentTok{## [1] 12.87}
\CommentTok{## }
\CommentTok{## $DIVC}
\CommentTok{## [1] 9.1}
\CommentTok{## }
\CommentTok{## $MLNP}
\CommentTok{## [1] 4.23}
\CommentTok{## }
\CommentTok{## $DIVNP}
\CommentTok{## [1] 4.01}
\CommentTok{## }
\CommentTok{## $NP_C}
\CommentTok{## [1] 2.26}
\end{Highlighting}
\end{Shaded}

\hypertarget{references}{%
\section*{References}\label{references}}
\addcontentsline{toc}{section}{References}

\hypertarget{refs}{}
\leavevmode\hypertarget{ref-Cooper1976}{}%
Cooper, T. C. (1976). Measuring written syntactic patterns of second
language learners of German. \emph{Journal of Educational Research},
\emph{69}(5), 176--183.
\url{https://doi.org/10.1080/00220671.1976.10884868}

\leavevmode\hypertarget{ref-Csardi2006}{}%
Csardi, G., \& Nepusz, T. (2006). The igraph software package for
complex network research. \emph{InterJournal (Complex Systems)}.

\leavevmode\hypertarget{ref-Denis2012}{}%
Denis, P., \& Sagot, B. (2012). Coupling an annotated corpus and a
lexicon for state-of-the-art POS tagging. \emph{Language Resources and
Evaluation}, \emph{46}(4), 721--736.
\url{https://doi.org/10.1007/s10579-012-9193-0}

\leavevmode\hypertarget{ref-Green2011}{}%
Green, N. (2011). Dependency Parsing. \emph{WDS'11 proceedings of
contributed papers, part i}. Prague.

\leavevmode\hypertarget{ref-Hunt1965}{}%
Hunt, K. (1965). \emph{Grammatical structures written at three grade
levels}. Champaign, IL: NCTE.

\leavevmode\hypertarget{ref-Hunt1970}{}%
Hunt, K. (1970). Do sentences in the second language grow like those in
the first? \emph{TESOL Quarterly1}, \emph{4}(3), 195--202.

\leavevmode\hypertarget{ref-Lu2010}{}%
Lu, X. (2010). Automatic analysis of syntactic complexity in second
language writing. \emph{International Journal of Corpus Linguistics},
\emph{15}(4), 474--496. \url{https://doi.org/10.1075/ijcl.15.4.02lu}

\leavevmode\hypertarget{ref-Nivre2006}{}%
Nivre, J., Hall, J., \& Nilsson, J. (2006). MaltParser : A data-driven
parser-generator for dependency parsing. \emph{LREC 2006}, 2216--2219.

\leavevmode\hypertarget{ref-Vandeweerd2021b}{}%
Vandeweerd, N., Housen, A., \& Paquot, M. (n.d.). Applying
phraseological complexity measures to L2 French: A partial replication
study. \emph{International Journal of Learner Corpus Research}.

\end{document}
